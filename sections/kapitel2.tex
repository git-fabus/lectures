\section{Fehleranalyse}
\subsection{Rechnerarithmetik}
\begin{example}
Betrachte $F=F(10,5,-4,5)$ und Maschinenzahlen
\begin{align*}
x=2.5684 \cdot 10^{0}= 2.56840000 \\
y=3.2791 \cdot 10^{-3} = 0.0032791
\end{align*}
Es gilt:\\
\begin{center}
$\begin{rcases}
	x+y= 2.5716792 \\
	x-y= 2.5651209 \\
	x \cdot y= 0.00842204044 \\
	\frac{x}{y}= 783.2637004
\end{rcases} \not\in F$
\end{center}
\end{example}
\begin{remark}
Die Menge $F(b,t,e_{min},e_{max})$ ist nicht abgeschlossen bezüglich der Grundrechenarten und können somit im Allgemeinen nicht im Computer implementiert werden.
\end{remark}
\paragraph{Lösung}
Wir runden das Ergebnis und implementieren so eine Pseudoarithmetik.
Das bedeutet, wir ersetzen $\circ \in  \{+,-,\cdot, \div\}$ durch $\boxcircle \in \{\boxplus, \boxminus, \boxdot, \boxslash\}$ definiert durch:
\begin{equation}\label{eqn:rundung}
x \boxcircle y \coloneqq rd (x \circ y)
\end{equation}
Auf Hardwareebene wird üblicherweise mit einer längeren Matisse gearbeitet und dann normalisiert und gerundet. Dies entspricht dem IEEE 754 Standard.
\begin{remark}
Für $|x|,|y|, |x \circ y| \in [z_{min},z_{max}]$ impliziert \eqref{eqn:rundung}, dass
\[
	\frac{|x \boxcircle y - x \circ y|}{|x \circ y|}= \frac{rd(x \circ y - x \circ y}{|x \circ y} \le \varepsilon_{mach}
\]

Das bedeutet, dass $\boxcircle$ im Computer bestmöglich umgesetzt ist.
\end{remark}
\begin{example}
Betrachte $F=F(10,5,-4,5)$ \\
\begin{itemize}
	\item Setze:
\begin{itemize}
	\item $a=0.98765$
	\item $b=0.012424$
	\item $c=-0.0065432$
\end{itemize}
Dann gilt:
\[
	(a+b)+c=a+(b+c)=0.9925208
\]
Numerisch gilt:
\[
(0.98765 \boxplus 0.012424)\boxminus 0.0065432= rd(0.9935568)=0.99356
\]
und
\[
0.98765 \boxplus (0.012424\boxminus 0.0065432)=rd(0.9935308)=0.99353
\]
\item Setze
	\begin{itemize}
		\item $a=4.2832$
		\item $b=-4.2821$
		\item $5.7632$
	\end{itemize}
Dann gilt
\[
(a+b)\cdot c = a \cdot c + b \cdot c = 0.006339520000001
\]
Numerisch gilt:
\end{itemize}
\end{example}
\begin{remark}
Mathematisch äquivalente Algorithmen auf Fließkommazahlen können je nach Implementierung zu wesentlich unterschiedlichen Ergebnissen führen, selbst wenn die Eingangszahlen exakt dargestellt werden.
\end{remark}
\subsection{Auslöschung}
Unglücklicherweise pflanzen sich numerische Fehler, zum Beispiel durch Rundung im Verlauf eines Algorithmus fort.

%Hier fehlt etwas

\paragraph{Achtung:} Ist $|x+-y|$ wesentlich kleiner als $|x|$ oder $|y|$, kann der relative Fehler massiv verstärkt werden. 
Dieses Phänomen heißt Auslöschung.\\
Bei der Konstruktion von Algorithmen sollte Auslöschung möglichst vermieden werden.


