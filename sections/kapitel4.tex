\section{Das Sortierproblem}
\paragraph{Gegeben} $n \in \N$ verschiedene Zahlen $z_1,\ldots,z_n \in \R$.
\paragraph{Gesucht} Permutation $\pi_1,\ldots,\pi_n$, so dass $z_{\pi_1} < \ldots < z_{\pi_n}$
\begin{definition}[Permutation]
Eine Permutation $\pi$ von $\{1,2,\ldots, n\}$ ist eine bijektive Abbildung von $\{1,2,\ldots,n\} $ auf sich selbst. Wir schreiben $\pi(k)=\pi_k$ für $k=1,\ldots,n$
\end{definition}
\begin{remark}
Da wir die Zahlen $z_1,\ldots,z_n$als verschieden annehmen ist das Sortierproblem eindeutig lösbar.
\end{remark}
\begin{algorithm}
\label{alg:bruteforce}
\caption{Brute-Force}
Probiere so lange alle möglichen Permutationen durch, bis die gewünschte Sortierung vorliegt
\end{algorithm}
\begin{theorem}
Es gibt $n! = n(n-1)\ldots 2 \cdot 1$ Permutationen der Menge $\{1,2,\ldots,n\} $
\end{theorem}
\begin{proof}
Wir haben 
\begin{itemize}
	\item $n$ Möglichkeiten die erste Zahl auszuwählen
	\item $n-1$ Möglichkeiten die zweite Zahl auszuwählen
	\item \ldots
	\item 1 Möglichkeit die letzte Zahl auszuwählen
\end{itemize}
woraus die Behauptung folgt.
\end{proof}
Im schlimmsten Fall (worst case) muss der Algorithmus \ref{alg:bruteforce} also
\[
	(n-1)\cdot n!
\]
Vergleiche durchführen. Da dies extrem aufwändig sein kann, betrachten wir im Folgenden einem Algorithmus, der die transitive Struktur
\[
x<y \text{ und } y<z \implies x<z
\]
der Ordnungsrelation ausnutzt.
