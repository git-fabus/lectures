\section{Kondition linearer Gleichungssysteme}
\begin{recall}
Die Kondition beschreibt, wie sehr Fehler in den Eingangsdaten eines Problems
verstärkt oder gedämpft werden und sich auf einen Fehler der Ausgangsdaten übertragen.
\end{recall}
Zum Lösen von $Ax=b$, $A$ invertierbar, bezeichnen wir $\Delta b$ als einen Eingangsfehler in $b$. 
\begin{align*}
&\implies x+ \Delta x = A^{-1}(b+\Delta b) = A^{-1}b +A^{-1} \Delta b \\
&\implies \Delta x= A^{-1} \Delta b
\end{align*}
Für ein verträgliches Matrix-Vektorraum-Paar gilt dann: 
\[
\|\Delta x\|= \|A^{-1} \Delta b\| \le \|A^{-1}\|\|\Delta b\| 
\]
und 
\[
	\frac{\|\Delta x\|}{\|x\|} \le \|A^{-1}\| \frac{\|\Delta b\|}{\|x\|}\le  \|A^{-1}\| \|A\| \frac{\|\Delta b\|}{\|b\|}
\]
