\section{Miller-Algorithmus}
Betrachte die Rückwärtsrekursion zu
\[
q_k=a_kq_{k-1} + b_k q_{k-2} , k=2,3,\ldots
\]
d.h.
\[
q_{k-2}= -\frac{a_k}{b_k}q_{k-1} + \frac{1}{b_k}q_k , k=n,n-1,\ldots,2
\]
mit Startwerten $q_n=0, q_{n-1}=1$. Für $a_k=a, b_k=b$ besitzt das charakteristische Polynom
\[
q(\mu)=\mu^2+ \frac{a}{b}\mu -\frac{1}{b}
\]
Die Nullstellen:
\[
\mu_{1,2}=\frac{-a\pm\sqrt[2]{a^2+4b}}{2b}=\frac{1}{\lambda_{1,2}}
\]
\paragraph{Idee:} Wende Satz \ref{thm:nst-polynom} "rückwärts" an. \\
Lösen wir
\[
\alpha + \beta = 0
\]
und
\[
\alpha\mu_1+\beta\mu_2=q_{n-1}=1
\]
ergibt sich:
\[
\alpha= \frac{\lambda_1\lambda_2}{\lambda_2-\lambda_1} \neq 0
\]
\[
\beta = -\frac{\lambda_1\lambda_2}{\lambda_2-\lambda_1}\neq 0
\]
und
\[
q_k= \alpha \mu_1^{n-k}+ \beta \mu_2^{n-k}= \frac{\alpha}{\lambda_1^{n-k}}+\frac{\beta}{\lambda_2^{n-k}}
\]
Für $|\lambda_1| < |\lambda_2|$ folgt, dass
\begin{align*}
	p_k^{\left( n \right)}
	&\coloneqq \frac{q_k}{q_0} \\
	&= \frac{\frac{\alpha}{\lambda_1^{n-k}}+\frac{\beta}{\lambda_2^{n-k}}}{\frac{\alpha}{\lambda_1^{n}}+\frac{\beta}{\lambda_2^{n}}} \\
	&=\frac{\lambda_1^{k}+\frac{\beta}{\alpha}\lambda_2^{k}\left( \frac{\lambda_1}{\lambda_2} \right)^{n}}{1+\frac{\beta}{\alpha}\left( \frac{\lambda_1}{\lambda_2} \right)^{n}}\\
	&\to \lambda_1^{k}=p_k
\end{align*}
\paragraph{Beobachtung:} Für großes n approximiert $p_k^{(n)}$ die Minimallösung.\\
\begin{algorithm}[H]
\caption{Miller-Algorithmus}
 \KwData{n genügend groß $\{a_k\}_{k=2}^{n}, \{b_k\}_{k=2}^{n} $}
 \KwResult{Approximation von $\{p_k^{(n)}\}_{k=0}^{n}$ um eine normierte Minimallösung}
 Setze $\hat{p_n}=0, \hat{p_{n-1}} =1$\\
 \For{$k \gets n$ \KwTo $2$}{
$\hat{p_{k-2}}=-\frac{a_k}{b_k}\hat{p_{k-1}}+\frac{1}{b_k}\hat{p_k}$
 }
\For{$k \gets 0$ \KwTo $n$}{
$p_k^{\left( n \right)}=\frac{\hat{p_n}}{\sqrt[]{\hat{p_0^2}+\hat{p_n^2}}}$
}
\end{algorithm}
\begin{theorem}[Miller-Algorithmus]
Sei $\{p_k\}_{k=0}^{\infty}$ Minimallösung der normierten, homogenen Dreitermrekursion. %Vielleicht noch Reflin einfügen
Dann gilt für die Lösung des Miller-Algorithmus:
\[
\lim_{n \to \infty} p_k^{(n)}=p_k , k=0,1,2,\ldots
\]
\end{theorem}
\begin{proof}
Wir bemerken, dass sich jede Lösung der Dreitermrekursion als Linearkombination einer Minimallösung und einer dominanten Lösung schreiben lässt (Übung). Mit den richtigen $\alpha$ und $\beta$ folgt (Übung):
\begin{align*}
\hat{p_k} 
&= \alpha p_k + \beta q_k \\
&= \frac{p_kq_n-q_kp_n}{p_{n-1}q_n-q_{n-1}p_n}
&= \frac{q_n}{p_{n-1}q_n-q_{n-1}p_n}\left( p_k-\frac{p_n}{q_n}q_k \right)
\end{align*}
Aus (Normierungs-Konstante)
\begin{align*}
\hat{p_0^2}+\hat{p_1^2}
&= \frac{q_n^2}{\left( p_{n-1} q_n- q_{n-1} p_n \right)^2}\left( p_0^2+p_1^2 -2 \frac{p_n}{q_n}(p_0q_0+p_1q_1) + \frac{p_n^2}{q_n^2}(q_0^2 +q_1^2) \right) \\
&= \frac{q_n^2}{(p_{n-1}q_n-q_{n-1}p_n)^2}\left( 1- 2 \frac{p_n}{q_n}+\frac{p_n^2}{q_n^2} \right)
\end{align*}
folgt:
\begin{align*}
p_k^{(n)}
&= \frac{\hat{p_k}}{\sqrt{\hat{p_0^2}+ \hat{p_1^2}}} \\
&= \frac{p_k-\frac{p_n}{q_n}q_k}{\sqrt{1-2 \frac{p_n}{q_n}+ \frac{p_n^2}{q_n^2}}}
&\to p_k
\end{align*}
\end{proof}
Wir können also auch für schlecht konditionierte Probleme stabile Algorithmen finden.
