\section*{Motivation}
Für $\K \in \{\R, \C \}$ bezeichnen wir $\K^{n}$ den Vektorraum der  $\K$-wertigen Vektoren 
\[
\vec{x}=\begin{pmatrix} x_1 \\ \vdots \\ x_n \end{pmatrix}, x_{i}\in \R, i=1,\ldots,n
\]
und mit $\K^{m\times n}$ den Vektorraum der $\K$-wertigen Matrizen 
\[
A= \begin{bmatrix}
	a_{1,1} & \ldots & a_{1,n} \\

	\vdots & & \vdots \\
	a_{m,1} & \ldots & a_{m,n}
\end{bmatrix}, a_{i,j} \in \K, i=1,\ldots,m , j=1,\ldots,n
\]
In vielen Anwendungen in Physik, Ökonomie, Life Sciences, Informatik, etc.
müssen lineare Gleichungssysteme gelöst werden, also System der Form:
\[
Ax=b, A \in \R^{n\times n}, x,b \in \R^{n}
\]
Hierbei ist n oft sehr groß.
