\section{Vorzeichen-Betrag-Darstellung}
Um auch Zahlen mit Vorzeichen am Computer darstellen zu können, betrachten wir im Folgenden die Vorzeichen-Betrag-Darstellung für Binärzahlen.
Das Binäralphabet besteht nur aus $0$ und $1$, welche wir auch als \emph{Bits} bezeichnen.
Bei einer Wortlänge von n Bits wir das erste Bit als Vorzeichen verwendet, die restlichen $n-1$-Bits für den Betrag der Zahl. Da die 0 die Darstellung $+0$ und $-O$ besitzt, können wir insgesamt $2^{n}-1$ Zahlen darstellen.

\begin{example}
Für $n=3$\\

\begin{table}[htpb]
	\centering
	\begin{tabular}{c c}
		Bitmuster & Dezimaldarstellung \\
		$000$ & $+0$ \\
		$001$ & $+1$ \\
		$\ldots$ & $\ldots$ \\
		$100$ & $-0$ \\
		$\ldots$ & $\ldots$ \\
		$111$ & $-3$
	\end{tabular}
\end{table}
\end{example}

\paragraph{Aber:} Diese Darstellung am Computer ist unpraktisch, da die vier Grundrechenarten auf Hardwareebene typischerweise mit Hilfe von Addition und Zusatz-Logik umgesetzt werden.
\paragraph{Lösung:} Komplementdarstellung
