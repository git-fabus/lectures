\section{Komplementdarstellung}

\begin{definition}[(b-1)-Komplement]
Sei $z=(z_{n-1}\ldots z_1 z_0)_b$ eine n-stellige b-adische Zahl. Das \emph{(b-1)-Komplement} $K_{b-1}(z)$ ist definiert als:
\[
K_{b-1} = (b-1-z_{n-1},\ldots, b-1-z_0)_b
\]
\end{definition}
Geben wir hierzu direkt ein paar Beispiele an
\begin{example} Komplemente \\
\begin{itemize}
	\item $K_9((325)_{10})= (674)_{10}$  (9er-Komplement im 10-er System)
	\item $K_1((10110)_2)=(01001)_2$ (1er-Komplement im 2er System)
\end{itemize}
\end{example}
\begin{definition}[b-Komplement]
Das b-Komplement einer b-adischen Zahl $z\neq 0$ ist definiert als \[
K_b(z)=K_{b-1}(z) +(1)_b
\]
\end{definition}
\begin{example}
\begin{itemize}
	\item $K_{10}((325)_{10}) = (674)_{10} + (1)_{10} = (675)_{10}$
\end{itemize}
\end{example}
\begin{lemma}[]
Für jede n-stellige b-adische Zahl $z$ gilt:
\begin{itemize}
	
	\item i) $z+K_{b-1}(z)=(b-1,\ldots,b-1)_b = b^{n}-1$
	\item ii) $K_{b-1}\left( K_{b-1}(z) \right)=z$
\end{itemize}
Ist außerdem $z\neq 0$ so gilt:
\begin{itemize}
	
	\item iii) $z+K_b(z)=b^{n}$
	\item iv) $K_b(K_b(z))=z$
\end{itemize}
\end{lemma}
\begin{proof}
\begin{itemize}Hilfssatz \\
	\item (i) Durch nachrechnen:
		\begin{align*}
		z+K_{b-1}(z)
		&=(z_{n-1} \ldots z_0)_b +(b-1-z_{n-1}, \ldots, b-1-z_0)_b \\
		&=\sum_{i=0}^{n-1}z_ib^{i}+ \sum_{i=0}^{n-1}(b-1-z_i)b^{i}\\
		&= \sum_{i=0}^{n-1}(b-1)b^{i} = (b-1, \ldots, b-1)_b \\
		&= (b-1)\sum_{i=0}^{n-1}b^{i} \\
		&=(b-1)\left( \frac{b^{n}-1}{b-1} \right) \\
		&= b^{n}-1
		\end{align*}
	\item (ii) per Definition
	\item (iii) Nachrechnen:
	\begin{align*}
		z+K_b(z) = z+K_{b-1} + 1 = b^{n}-1+1=b^{n}
	\end{align*}
	\item Definiere $\hat{z} = K_b\left( z \right)= K_{b-1}(z) + (1)_b > 0$ und rechne
		\[
		z+K_b(z)=b^{n}=\hat{z} +K_b{\hat{z}}+ K_b\left( K_b(z) \right) \implies \text{Behauptung.}
		\]
\end{itemize}



\end{proof}

\begin{remark} Modifikation 
\begin{itemize}
	\item Die 3. Aussage gilt auch für $z=0$, falls man dann bei der Addition von 1 die Anzahl der Stellen erweitert.
	\item die 4. Aussage gilt für $z=0$, falls überall im Beweis modulo $b^{n}$ gerechnet wird.
\end{itemize}
\end{remark}
Außerdem impliziert die 3. Aussage des Lemmas, dass 
\begin{equation}\label{eqn:komplement}
	K_b(z)=b^{n}-z \text{.}
\end{equation}
Dies können wir geschickt zum Darstellen der $b^{n}$ verschiedenen ganzen Zahlen $z$ mit 
\[
-\left\lfloor \frac{b^{n}}{2}\right\rfloor \le z \le \left\lceil \frac{b^{n}}{2}\right\rceil -1
\]
nutzen. Diesen Bereich nennt man darstellbaren Bereich.
\begin{definition}[b-Komplement-Darstellung]
	Die b-Komplement-Darstellung $(z)_{K_b} = (z_{n-1} \ldots z_0)_b$ einer Zahl $z \in  \Z$ im darstellbaren Bereich ist definiert als:
	\[
		(z)_{K_b}= \begin{cases}
			(z)_b & \text{falls } z \ge 0 \\
			\left( K_b(|z|) \right)_b & \text{falls } z<0
		\end{cases}
	\]
\end{definition}

\begin{example} Der darstellbare Bereich.
\begin{itemize}
	\item Sei $b = 10$, $n=2$. \\ 
Dann impliziert \eqref{eqn:komplement}, dass
\[
K_{10}(50)= 10^2-50=50
\]
\[
K_{10}(49)= 100 -49=51 \text{.}
\]
Der darstellbare Bereich ist nun 
\[
-50 \le z \le 49
\]
und hat konkrete Darstellungen:

\begin{center}
\begin{tabular}{ c c }
 Darstellung & Zahl   \\
 $0$ & $+0$  \\
 $1$ & $+1$ \\
 \ldots & \ldots \\
 $49$ & $+49$ \\
 $50$ & $-50$ \\
 \ldots & \ldots \\
 $99$ & $-1$
\end{tabular}
\end{center}

\item Sei $b=2$, $n=3$ Der darstellbare Bereich ist $-4 \le  z \le 3$. \\
\begin{center}
\begin{tabular}{ c c }
 Bitmuster & Dezimaldarstellung  \\ 
 $000$ & $0$   \\  
 $001$ & $1$  \\
 \ldots & \ldots \\
 $100$ & $-4$ \\
 \ldots & \ldots \\
 $111$ & $-1$
\end{tabular}
\end{center}
\end{itemize}
\end{example}%
Wir betrachten nun Addition und Subtraktion zweier Zahlen in b-Komplement-Darstellung. 
Hierzu bezeichne $(x)_{K_b} \oplus (y)_{K_b}$ die ziffernweise Addition der Darstellungen von x und y mit Übertrag ("schriftlich rechnen"), 
wobei ein eventueller Überlauf auf die $(n+1)$-te Stelle vernachlässigt wird (wir rechen also immer mit modulo $b^{n}$.
\begin{theorem}[Addition in b-Komplement-Darstelllung]
	Seien x und y zwei n-stellige, b-adische Zahlen und x,y und $x+y$ im darstellbaren Bereich.
Dann gilt:
\[
(x+y)_{K_b} = (x)_{K_b} \oplus (y)_{K_b}
\]
\end{theorem}
\begin{proof}
Wir betrachten dazu mehrere Fälle.
\begin{itemize}
	\item Fall $x,y \ge 0$: \\
\begin{align*}
(x)_{K_b} \oplus (y)_{K_b}
    &\overset{\text{Def.}}{=}\left( (x)_b + (y)_b \right) \mod b^{n} \\
    &\overset{\text{Def.}}{=} (x+y) \mod b^{n} \\
    &\overset{\text{Def.}}{=} (x+y)_{K_b} \\
\end{align*}
Der letzte Schritt ist möglich, da $(x+y)_{K_b}$ im darstellbaren Bereich sind.
\item Fall $x,y <0$
\begin{align*}
(x)_{K_b} \oplus (y)_{K_b}
&\overset{Def.}{=} \left( \left( K_b(|x|) \right)_b + \left( K_b(|y|) \right)_b \right) \mod b^{n} \\
&\overset{   }{=}\left( (K_b(|x|)) +\left( K_b\left( |x| \right) \right)\right) \mod b^{n} \\
&= (b^{n}-|x| + b^{n}-|y|) \mod b^{n} \\
&= (x+y) \mod b^{n} \\
&= (x+y)_{K_b}
\end{align*}
\item Fall $x\ge 0, y<0$: \\
\begin{align*}
(x)_{K_b} \oplus (y)_{K_b}
&\overset{Def.}{=} ((x)_b + (K_b(|y|))_b) \mod b^{n} \\
&=(x+K_b(|y|)) \mod b^{n} \\
&=(x+b^{n}-|y|) \mod b^{n} \\
&=(x+y) \mod b^{n} \\
&= (x+y)_{K_b}
\end{align*}
\item Fall $x<0, y\ge 0$: Analog
\end{itemize}
\end{proof}
\begin{theorem}[Subtraktion in b-Komplement-Darstellung]
	Seien x und y n-stellige b-adische Zahlen und x,y und x-y im darstellbaren Bereich.
	Dann gilt:
	\[
	(x-y)_{K_b} = (x)_{K_b} \oplus \left( K_b(y) \right)_{K_b}
	\]
\end{theorem}
\begin{proof}
\begin{itemize}
	\item Fall $y=0$ nichts zu zeigen.
	\item $y \neq 0$: \eqref{eqn:komplement} impliziert:
		\[
		-y=K_b(y) -b^{n}
		\]
	mit modulo $b^{n}$-rechnen folgt, dass
	\[
		(-y)_{K_b}= (K_b(y))_{K_b}
	\]
	ist und somit:
	\begin{align*}
		(x-y)_{K_b}
		&= (x+(-y))_{K_b} \\
		&= (x)_{K_b} \oplus (-y)_{K_b} \\
		&= (x)_{K_b} \oplus (K_b(y))_{K_b}	
	\end{align*}
\end{itemize}
\end{proof}
\begin{example}
Für $b=10$ und $n=2$ ist der darstellbare Bereich $-50 \le  z \le 49$
\begin{itemize}
	\item $(20+7)_{K_{10}} = (20)_{K_{10}} \oplus (7)_{K_{10}} =(27)_{K_{10}}=27$
	\item $28-5 = 28+ (-5)= (28)_{K_{10}} \oplus (95)_{K_{10}} = (23)_{K_{10}} =23 $
	\item $-18-20= (-18)+(-20) = \ldots = -38$
\end{itemize}
\end{example}
Die darstellbaren Zahlen kann man sich beim $K_b$-Komplement als Zahlenrad vorstellen.

%Hier muss eine coole Uhr eingefügt werden.

\paragraph{Achtung} Ein eventueller Überlauf bzw. Unterlauf wird im Allgemeinen nicht aufgefangen.
\begin{example}
In n-stelliger Binärarithmetik ist die größte darstellbare Zahl $x_{max} = (011\ldots 1)_{K_2}$ gleich  $2^{n-1}-1$. Hingegen ist $x_{max}+1=(100\ldots 0)_{K_2}$ und wir als $-2^{n-1}$ interpretiert.
\end{example}
