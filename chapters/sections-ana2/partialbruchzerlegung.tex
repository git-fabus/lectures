\section{Partialbruchzerlegung}
\begin{lemma}[Fundamentalsatz der Algebra]
Eigentlich ein starker Satz in der Algebra, wir verwenden ihn jedoch nur als Hilfssatz.

\begin{itemize}
	\item Sei $g$ ein Polynom in $\C$  von Grad n. Dann $\exists_{w_1,\ldots,w_k,a \in \C} $ und $v_1,\ldots,v_k \in \N$ sowie $\sum_{l>1}^{k}v_l=n, (w_l \neq w_j)$ mit:
		\[
			g(z) = a \prod_{l=1}^{k} (z-w_l)^{v_k}
		\]
		Wobei $(z-w_l)$ die Nullstelle und $v_k$ die Vielfachheit ist.   
	\item Falls Koeffizienten von $g$ reel, dann ist $w_l$ und auch $\overline{w_l}$ eine Nullstelle. 
\end{itemize}
$\implies \frac{1}{a}g$ ist Produkt quadratischer Polynome:
\[
	(z-w_l)(z-\overline{w_l})= z^2 -2Re(w_l)z + |w_l|^2
\]
\end{lemma}
\begin{lemma}[Abspaltung von Haptteilen]
Sei $r=\frac{f}{g}$ eine rationale Funktion, wobei $f,g$ Polynome in $\C$, mit n-fachen Pol in $w \in \C$.\\
Ohne Beschränkung der Allgemeinheit $f(w)\neq 0$ und
\begin{equation}
	\label{eqn:abspaltung}
	r(z) = \frac{f(z)}{(z-w)^n \cdot h(z)}
\end{equation}
mit $h(w)\neq 0$. \\rationale Funktion ohne Pol in $w$:
\[
	H(z) = \frac{a_n}{(z-w)^n}+ \frac{a_{n-1}}{(z-w)^{n-1}}+\ldots+\frac{a_1}{z-w}
\]
mit $a_n\neq 0$. Dabei sind $a_n,\ldots,a_1$ sowie $r_0$ eindeutig. 
\end{lemma}
\begin{proof}
	Aus \eqref{eqn:abspaltung} folgt:
	\[
	\frac{f(z)}{h(z)}-\frac{f(w)}{h(w)}=\frac{f(z)h(w)-h(z)f(w)}{h(z)h(w)}= \frac{(z-w)p(z}{h(z)}
	\]
	da $z \mapsto f(z)h(w)-h(z)f(w)$ Polynom mit Nullstellen in $w$ ist, $\exists_{\text{Polynom } p}$. 
Nun können wir \eqref{eqn:abspaltung} verwenden und es folgt:
\begin{align*}
	r(z)&= \frac{f(z)}{(z-a)^n \cdot h}= \frac{1}{(z-a)^n}\frac{f(w)}{h(w)}+\frac{1}{(z-w)^{n-1}} \frac{p(z)}{h(z)} \\
	    &=\frac{a_n}{(z-w)^n} + \frac{p(z)}{(z-w)^{n-1}h(z)}
\end{align*}
Nun führen wir einen Induktionsbeweis nach $n$.
\begin{itemize}
	\item \underline{IA $n=0$}:
		\[
		r=r_0
		\]
	\item \underline{IS $n-1 \to n$}:
		\[
			\implies r(z) = \frac{a_n}{(z-w)^n}+ \left( \frac{a_{n-1}}{(z-w)^{n+1}}+\ldots + \frac{a_1}{z-w} \right)
		\]
		Dies folgt direkt nach Induktionsannahme.
\end{itemize}
\end{proof}
\begin{theorem}[Satz der Partialbruchzerlegung]
	Sei $r=\frac{f}{g}$ rational mit Polynom $f,g$ und $g(z))= \prod_{l=1}^{k} (z-w_l)^{v_l} $ mit paarweise verschiedenen $w_l \in \C$. Dann gilt:
	\[
	r= \sum_{l=1}^{k}H_l +q
	\]
mit Hauptteilen $H_l$ in $w_l$ und Polynom $q$.   

\end{theorem}
\begin{proof}
Iterative Abspaltung von Hauptteilen liefert:
\begin{align*}
	H = H_1+r_1 &= H_1+H_2+r_{2} \\
	 &= \ldots \\
	 &= \sum_{l=1}^{k}H_l +r_k
\end{align*}
mit $r_k$ rational. Wir wissen:
\[
	\{\text{Pole von } r_{l+1}\} \subset \{\text{Pole von }r_l\} \setminus \{w_{l+1}\} 
\]
Dies impliziert:
\begin{itemize}
	\item $r_k$ hat keinen Pol
	\item $r_k$ ist Polynom.
\end{itemize}
\end{proof}
\begin{corollary}
	$\int r dx = \int q dx + \sum_{l=1}^{k}\sum_{j=1}^{v_l}\frac{a_{l,j}}{(x-w_l)^j}dx$ 
\end{corollary}
\noindent Dabei ist:
\begin{itemize}
	\item $\int q dx$ klar. 
	\item $\int_{s}^{t} \frac{1}{(x-w)^j}dx$ mit $j>1$ gleich: $ -\frac{1}{j-1}\cdot \frac{1}{(x-w)^{j-1}}\big|_{s}^{t}$  
	\item Nicht trivial: $\int_s^t \frac{a_1}{x-w}dx$. \\
		Hierbei gilt mit $w$ taucht auch $\overline{w}$ auf (mit Zähler $\overline{a_1}$ ). Das heißt zu berechnen ist:
\begin{align*}
	\int_s^t \left[ \frac{a}{x-w}+ \frac{\overline{a}}{x-\overline{w}}\right] dx 
	&= \int_s^t \frac{2Re(a)x-2Re(a\overline{w}}{x^2 - 2Re(w)x+|w|^2}dx \\
	&= \int_s^t \frac{Bx+c}{x^2+2bx+c}dx \text{ mit } b^2 <c \\
	&= \frac{B}{2}\int_s^t \frac{2x+2b}{x^2+2bx+c}dx + (c-Bb) \int_s^t \frac{1}{x^2+2bx+c}dx 
\end{align*}
Wir können nun getrennt beide Integrale ausrechnen:
\[
= \frac{B}{2}\log(x^2+2bx+c)\big|_s^t + \ldots
\]
und das etwas schwierigere Integral:
\[
	\int_s^t \frac{1}{x^2+2b+c}dx= \int_{\frac{s+b}{\sqrt[2]{c-b^2}}}^{\frac{t+b}{\sqrt[2]{c-b^2} }} \frac{1}{y^2+1}dy = \frac{1}{\sqrt[2]{c-b^2}} \arctan(y) \big|_{\frac{s+b}{\sqrt[2]{c-b^2} }}^{\frac{t+b}{\sqrt[2]{c-b^2} }}
\]
\end{itemize}
Als Übung kann das Integral vollständig ausgeschrieben werden. Dies sei dem Leser überlassen.
