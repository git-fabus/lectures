\section{Zufällige Ereignisse}
Die "Zutaten" für ein zufälliges Ereignis sind:
\begin{itemize}
	\item Zufallssituation, die (gedanklich) beliebig oft wiederholt werden kann
	\item Ein Versuch / eine Realisierung einer Zufallssituation
	\item Eine Menge aller möglichen Ereignisse
\end{itemize}
Dabei ist die Annahme, dass das Ergebnis jedes Versuch Element in der Ereignismenge ist.
\begin{definition}[(zufälliges) Ereignis]
Sei $\Omega$ eine \emph{Eregnismenge}. Ein \emph{(zufälliges) Ereignis} ist eine Teilmenge $A \subset \Omega$. Für ein Ergebnis eines Versuchs $\omega \in \Omega$ sagt man, dass $A$ eingetreten ist, falls $\omega \in A$.   
\end{definition}
\begin{remark}
$\Omega$ kann Ereignisse enthalten, die nie eintreten! Umgekehrt müssen alle eintretenden Ereignisse Teilmenge von $\Omega$ sein.
\end{remark}
\begin{example}
Typische Alltagssituationen sind Zufallssituationen: 
\begin{itemize}
	\item Werfen eines Würfels : $\Omega = \{1,\ldots,6\}$ \\
		Eine gerade Zahl würfeln: $A = \{2,4,6\}\subset \Omega$
	\item Lebensdauer eines Elektrogeräts: $\Omega= \{\omega \in \Omega | \omega \ge 0\}$ \\ 
		Lebensdauer beträgt 3-4 Jahre: $A = \{\omega \in \Omega | 3\le \omega \le 4\}$ 
	\item Überprüfen der Funktionsfähigkeit von Elektrogeräten: $\Omega= \{(\omega_1,\ldots,\omega_n) \in \Omega | w_i \in \{0,1\} \}$ \\
		Es funktionieren mindestens $k \in \N$ Geräte davon: 
		$A = \{(\omega_1,\ldots,\omega_n) \in \Omega | \sum_{i=1}^{n}w_i\ge k\}$ 
\end{itemize}
\end{example}
