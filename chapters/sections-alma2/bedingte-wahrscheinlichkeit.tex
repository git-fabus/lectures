\section{Bedingte Wahrscheinlichkeit}
Betrachte die Ereignisse $A,B, A \cap B$. \\
Absolute Häufigkeit: $h_n(A)$ \\
Relative Häufigkeit: $H_n(A)= \frac{h_n(A)}{n}$ \\
Wie oft tritt A ein, unter Bedingung, dass B schon eingetreten ist?
\begin{notation}
Wir schreiben $A|B$ für "A gegeben B"
\[
H_n(A|B) = \frac{h_n(A \cap B)}{h_n(B)} = \frac{H_n(A \cap B)}{H_n(B)}
\]
\end{notation}
\begin{definition}[bedingte Wahrscheinlichkeit]
	Sei $(\Omega, \mathcal{A}, P)$ ein Wahrscheinlichkeitsraum und $A,B \in \mathcal{A}$ mit $P(B)>0$. Die \emph{bedingte Wahrscheinlichkeit} von $A$ gegeben $B$ ist definiert als :
	\[
		P(A|B)= \frac{P(A \cap B)}{P(B)}
	\]
\end{definition}
\begin{example}
Würfle zwei mal mit einem idealisierten Würfel und betrachte die Ereignisse:
\begin{align*}
	A &= \{\text{Beim ersten Wurf wird eine 6 gewürfelt}\} \\
	B &= \{\text{die Summe beider Würfel ist 8}\} 
\end{align*}
Betrachte dazu die Tabelle:
\[
\implies P(B) = \frac{5}{36}, P(A \cap B)= \frac{1}{36}, P(A|B)= \frac{1}{5}
\]
Dies ist Konsistent zur Definition. 
\end{example}
Faustregel: Die Rechenregel für bedingte Wahrscheinlichkeit sind die für Wahrscheinlichkeit, falls das zweite Argument fest gelassen wird.
\begin{theorem}
	Sei $( \Omega, \mathcal{A}, P)$ ein Wahrscheinlichkeitsraum und $A,B,C \in \mathcal{A}$ mit $P(C) > 0$ Dann gilt:
	\begin{enumerate}
		\item $P(C|C)=1$
		\item $P(A|C)=1-P(\overline{A}|C)$
		\item $P(A \cup B |C)= P(A|C) + P(B|C) -P(A \cap B| C)$ 
	\end{enumerate}
\end{theorem}
\begin{proof}
Wir zeigen:
\begin{enumerate}
	\item $P(C|C)= \frac{P(C \cap C)}{P(C)}= \frac{P(C)}{P(C)}= 1$
	\item Beobachte: $A \cup \overline{A}= \Omega \implies A, \overline{A}$ unvereinbar.
		\[
		\implies (A \cap C) \cup (\overline{A} \cap C)=C
		\]
	Ebenfalls unvereinbar.
	\begin{align*}
		\implies 1 = \frac{P(C)}{P(C)} &= \frac{P((A \cap C) \cup (\overline{A} \cap C)}{P(C)} \\
					       &= \frac{P(A \cap C) + P(\overline{A}+C)}{P(C)} \\
					       &= P(A|C)+ P(\overline{A}|C)	
	\end{align*}
\item \begin{align*}
		P(A \cup B| C) &= \frac{P((A \cup B) \cap C)}{P(C)} \\
			       &= \frac{P((A \cap C) \cup (B \cap C)) \cup}{P(C)} \\
			       &= \frac{P(A \cap C) + P(B \cap C) - P((A \cap C) \cap (B \cap C))}{P(C)} \\
			       &= P(A |C ) + P(B|C) - P(A \cap B |C)
\end{align*}
\end{enumerate}
\end{proof}
\begin{example}[Stapelsuchproblem]
Durchsuche 7 gleich große Bücherstapel nach einem Buch. Die Wahrscheinlichkeit, dass das Buch überhaupt in den Stapeln ist, ist $0.8$.\\
Angenommen wir haben die ersten 6 Stapel durchsucht und das Buch nicht gefunden. Wie groß ist die Wahrscheinlichkeit, dass das Buch im letzten Stapel ist? \\
Ereignisse:
\[
A_i = \{\text{Buch ist im i-ten Stapel}\}, i = 1,\ldots,7 
\]
mit $P(A_i)=p \in (0,1)$ unbekannt. \\
Wir wissen: 
\begin{align*}
	0.8&= P(A_1 \cup \ldots \cup A_7) \\
	   &= P(A_1)+ \ldots + P(A_7) \\
	   &= 7p \\
	   &\implies p = \frac{0.8}{7} \approx 0.114 \\
\end{align*}
\begin{align*}
	P(A_7| \overline{A_1 \cup \ldots \cup A_6)}&= P(A_7 | \overline{A_1} \cap \ldots \cap \overline{A_6}) \\
						   &= \frac{P(A_7}{1-P(A_1 \cup \ldots \cup A_6} \\
						   &= \frac{p}{1-6p} \\
						   &\approx 0.3636
\end{align*}
\end{example}

