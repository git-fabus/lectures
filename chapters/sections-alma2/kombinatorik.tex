\section{Kombinatorik}
Wir bleiben bei endlichen Ereignismengen mit $|\Omega|= n \in \N$ Elementen. Solche Zufallsexperimente können als Urnen mit Kugeln (den Ereignissen) aufgefasst werden. Ein Elementarereignis entspricht den ziehen einer Kugel. \\
\paragraph{Varianten:}
\begin{itemize}
	\item Werden die entnommenen Kugeln vor dem nächsten Experiment wieder zurückgelegt?
	\item Ist die Reihenfolge der entnommenen Kugeln relevant?
		\begin{itemize}
			\item Falls ja: Variationen
			\item Sonst: Kombinationen
		\end{itemize}
\end{itemize}
\begin{theorem}
	Sei $\Omega$ eine Menge mit $|\Omega| = n \in \N$.
	\begin{enumerate}[label=\arabic*)]
		\item Die Anzahl aller Kombinationen der Länge $m$ mit zurücklegen ist:
			\[
				| \{\{\omega_1,\ldots,\omega_n\}| \omega_1,\ldots, \omega_n \in \Omega \}| = {n+m-1 \choose m} 
			\]
			wobei ${n \choose k} = \frac{n!}{(n-m)!m!}$ für $n \in \N, 0\le k <n$.
		\item Die Anzahl aller Kombinationen der Länge $m\le n$ ohne Zurücklegen ist:
			\[
				| \{\{\omega_1,\ldots,\omega_n\} | \omega_1, \ldots, \omega_n \in \Omega\} = {n \choose m}
			\]
	\end{enumerate}

\end{theorem}
\begin{proof}
Ohne Beschränkung der Allgemeinheit: Nehme an, dass die Elemente in $\Omega$ von 1 bis n durchnummeriert sind.Zähle Elemente in 
\[
\{(a_1,\ldots,a_n) | 1\le a_1\le a_2 \le \ldots \le a_m \le n\} = A 
\]
Definiere $b_i = a_i + i-1, i = 1, \ldots, m$. Dies definiert eine Bijektion auf $A$ mit Bild:
\[
\{(b_1,\ldots,b_m) | 1\le b_1\le \ldots\le b_m\le n+m-1\}=B 
\]
Zähle nun Elemente in B. Bei der Vernachlässigung der Sortierung sind das
\[
	\frac{(n+m-1)!}{(n-1)!}
\]
Elemente. Es interessiert uns nur eines dieser $m!$ Möglichkeiten, daraus folgt die Behauptung. \\
Der zweite Teil erfolgt Analog zu 1.
\end{proof}
\begin{example}[Geburtenparadoxon]
Bestimme die Wahrscheinlichkeit, dass zwei Personen im selben Raum am gleichen Tag Geburtstag haben. Dazu sei dies der Modellrahmen:
\begin{itemize}
	\item $n \in \N$ die Personen im Raum
	\item keine Geburtstage am 29. Februar
	\item Alle restlichen 365 Tage im Jahr sind als Geburtstage alle gleich wahrscheinlich.
\end{itemize}
Demnach handelt es sich um ein La-Place Modell. \\
Betrachtetes Ereignis:
\[
A = \{\text{mindesten 2 Personen haben am gleichen Tag Geburtstag}\} 
\]
Um diese Wahrscheinlichkeit zu bestimmen, betrachte man das Komplementärereignis:
\[
\overline{A}= \{\text{alle Personen haben an verschiedenen Tagen Geburtstag}\} 
\]
Wir können dies wie folgt bestimmen:
\[
P(\overline{A})=\frac{\text{Anzahl günstiger Ergebnisse}}{\text{Anzahl möglicher Ergebnisse}}
\]
Wir bestimmen nun das richtige "Urnenmodell":
\begin{itemize}
	\item Für $\overline{A}$ : Ziehen ohne zurücklegen und Beachtung der Reihenfolge.
	\item Für $\Omega$ : Ziehen mit zurücklegen und Beachtung der Reihenfolge.
\end{itemize}
Daraus ergibt sich:
\[
P(A)-P(\overline{A}) = 1- \frac{365\cdot 364\cdot \ldots\cdot (365-n+1)}{365^n}
\]
Wir betrachten ein paar Beispielwerte:
\begin{align*}
	n=23 &\implies P(A) \approx 0,507 \\
	n=70 &\implies P(A) \approx 0,999 \\
	n=160 &\implies P(A) \approx 1
\end{align*}
\end{example}
