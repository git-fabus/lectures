\section{Multiplikationsregeln}
Umstellen der Definition der bedingten Wahrscheinlichkeit liefert:
\[
P(A \cap B) = \begin{cases}
	P(A|B) \cdot P(B) \\
	P(B|A) \cdot P(A)
\end{cases}
\]
\begin{example}[Notenverteilung]
Es sei:
\begin{itemize}
	\item 70 Prozent aller Studierenden haben eine Note, die besser als 3 ist
	\item Davon sind 25 Prozent, die eine bessere Note als 2 haben.
\end{itemize}
Berechne die Wahrscheinlichkeit, dass eine beliebige Person in der Menge mit einer besseren Note als 2 abschließt. Ereignisse:
\begin{align*}
	A&= \{\text{Note} \le 2\} \\
	B&= \{\text{Note} \le 3\} 
\end{align*}
Nach der Multiplikationsregel:
\[
P(A)= P(A \cap B) = P(A|B) \cdot P(B) = 0.175
\]
\end{example}
\begin{theorem}[erweiterte Multiplikationsregel]
	Sei $(\Omega, \mathcal{A}, P$ ein Wahrscheinlichkeitsraum und $A_1,\ldots,A_n \in \mathcal{A}$ mit $P(A_1 \cap \ldots \cap A_{n-1}>0$. Dann giltː
	\[
		P(A_1 \cap \ldots \cap A_n) = P(A_1)\cdot P(A_2 |A_1)\cdot  P(A_3| A_1 \cap A_2) \ldots P(A_n| A_1 \cap \ldots \cap A_{n-1})
	\]
\end{theorem}
\begin{proof}
Induktion. % Seite 24 in V1G6
\end{proof}
Die erweiterte Multiplikationsregel erlaubt die Darstellung eines mehrstufigen Zufallsexperiment als Baumdiagramm.
\begin{example}[Ziegenproblem]
	Wähle eine aus drei Türen. Hinter einer der Türen befindet sich ein Auto, hinter den anderen beiden nur Ziegen. Nun öffnet ein anderer eine der Türen, in dem das Auto nicht steht. \\
	Man darf sich nun erneut entscheiden. Ist dies Vorteilhaft?
% Fancy Diagramme Seite 25 in V1G6
\end{example}

\begin{theorem}[Satz von der totalen Wahrscheinlichkeit]
    \label{thm:totale_wahrscheinlichkeit}
    Sei $(\Omega, \mc{A},P)$ ein Wahrscheinlichkeitsraum und $B_1,B_2,\ldots$ $A$ eine Zerlegung von $\Omega$ das heißt \[
        \bigcup_{i=1}^{\infty}B_{i} \Omega \text{ mit } B_i \cap B_j , i\neq j
    .\]    
    Dann gilt für ein beliebiges Ereignis $A \in \mc{A}$, dass \[
    P(A)=\sum_{i=1}^{\infty} P(A |B_i)P(B_i)
    .\]  
    Dabei werden alle Terme mit $P(B_i)=0$ weggelassen.
\end{theorem}
\begin{proof}
Rechne: \[
    A = A \cap \Omega = A \cap  (\bigcup_{i=1}^{\infty}B_i)= \bigcup_{i=1}^{\infty} (A \cap B_i)
.\]
Es gilt weiter:
\begin{align*}
    P(A)= P(\bigcup_{i=1}^{\infty}) &= \sum_{i=1}^{\infty} P(A \cap B_i) \\
                                    &= \sum_{i=1}^{\infty} P(A|B_i) P(B_i)
.\end{align*}
\end{proof}


\begin{theorem}
    [Bayesche Formel] Mit den gleichen Voraussetzungen wie in Satz \ref{thm:totale_wahrscheinlichkeit} gilt die Bayesche Formel: \[
    P(B_j|A)= \frac{P(A|B_j) P(B_j)}{P(A)}= \frac{P(A|B_j)P(B_j)}{\sum_{i=1}^{\infty} P(A|B_i)P(B_i)},j \in \N
    .\] 
\end{theorem}
\begin{proof}
Mittels der Multiplikationsregeln und dem Satz der totalen Wahrscheinlichkeit folgt die Behauptung direkt.
\end{proof}

\begin{example}
Betrachte die Statisk einer Netzbetreibers für Störungen innerhalb eines Jahres:
\begin{itemize}
    \item 50 \% aller Störungen sind an Switches 
    \item 30 \% aller Störungen sind an Routern 
    \item 20 \% aller Störungen sind nicht klar zugeordnet
\end{itemize}
Kurzfristig können gelöst werden:
\begin{itemize}
    \item 50 \% der Störungen an Switches 
    \item 30 \% der Störungen an Routern 
    \item 5 \% der Störungen von den nicht klar zugeordneten Störungen.
\end{itemize}
\paragraph{Wie viel Prozent aller Störungen können spontan gelöst werden?}
Wir betrachten die Ereignisse: 
\begin{align*}
    B_1 &= \{ \text{Störung an einem Switch}\} \\
    B_2 &= \{\text{ Störungen an einem Router }\} \\
    B_3 &= \{\text{sonstige Störung}\} 
.\end{align*}
Die Wahrscheinlichkeiten dazu sind:
\begin{align*}
    P(B_1)&= 0.5\\ P(B_2)&=0.3 \\ P(B_3)&=0.2
\end{align*} mit $B_1 \cup B_2 \cup B_3 = \Omega$. Außerdem ist $A = \{\text{Störung konnte spontan behoben werden}\} $. Nach dem Satz der totalen Wahrscheinlichkeit gilt:
\begin{align*}
    P(A) &= P(A|B_1)P(B_1)+P(A|B_2)P(B_2)+P(A|B_3)P(B_3)\\
         &= 0.5 \cdot 0.5 + 0.3 \cdot 0.3 + 0.05\cdot 0.2 \\
         &\implies 35 \text{\% aller Störungen konnten spontan Behoben werden}
.\end{align*}%Zweites Beispiel hinzufügen Seite 31
\end{example}

