\section{Rechnen mit Wahrscheinlichkeiten}
Sei $(\Omega, \mathcal{A}$ ein messbarer Raum und $A \subset \mathcal{A}$ ein Ereignis. Betrachte $n \in \N$  sich nicht beeinflussende, d.h. unabhängige Versuche.

\begin{definition}[Häufigkeiten]
	Wir unterscheiden in:
	\begin{itemize}
		\item Die \emph{absolute Häufigkeit} $h_n(A)$, welche an gibt, wie oft $A$ bei $n$ Versuchen eintritt.
		\item Die \emph{relative Häufigkeit}: $H_n(A) = \frac{h_n(A)}{n}$
	\end{itemize}
\end{definition}
\begin{theorem}
	Sei $(\Omega, \mathcal{A})$ ein messbarer Raum. Dann gilt:
	\begin{enumerate}
		\item $H_n(\Omega) \ge 0$ für alle $A \subset \mathcal{A}, n \in \N$
		\item $H_n(\Omega)=1$
		\item $H_n(A \cup B) = \frac{h_n(A)+h_n(B)}{n}= H_n(A)+H_n(B)$ 
	\end{enumerate}

\end{theorem}

