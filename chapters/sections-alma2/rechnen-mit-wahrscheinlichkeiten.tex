\section{Rechnen mit Wahrscheinlichkeiten}
Sei $(\Omega, \mathcal{A}$ ein messbarer Raum und $A \subset \mathcal{A}$ ein Ereignis. Betrachte $n \in \N$  sich nicht beeinflussende, d.h. unabhängige Versuche.

\begin{definition}[Häufigkeiten]
	Wir unterscheiden in:
	\begin{itemize}
		\item Die \emph{absolute Häufigkeit} $h_n(A)$, welche an gibt, wie oft $A$ bei $n$ Versuchen eintritt.
		\item Die \emph{relative Häufigkeit}: $H_n(A) = \frac{h_n(A)}{n}$
	\end{itemize}
\end{definition}
\begin{theorem}
	\label{thm:motivation-axiome}
	Sei $(\Omega, \mathcal{A})$ ein messbarer Raum. Dann gilt:
	\begin{enumerate}
		\item $H_n(\Omega) \ge 0$ für alle $A \subset \mathcal{A}, n \in \N$
		\item $H_n(\Omega)=1$
		\item $H_n(A \cup B) = \frac{h_n(A)+h_n(B)}{n}= H_n(A)+H_n(B)$ 
	\end{enumerate}

\end{theorem}
\begin{recall}
	$(\Omega, \mathcal{A})$ ist ein messbarer Raum. Dabei ist $\Omega$ die Ereignismenge und $\mathcal{A}$ eine $\sigma$-Algebra. Für diese müssen folgende Dinge erfüllt sein:
	\begin{enumerate}
		\item $\Omega \in \mathcal{A}$
		\item $A \in \mathcal{A} \implies \overline{A}\in \mathcal{A}$
		\item $A_i \in \mathcal{A}, i \in \N \implies \bigcup_{i=1}^{\infty} A_i \in \mathcal{A}$ 
	\end{enumerate}
Außerdem haben wir die absolute Häufigkeit $h_n(A)$ und die relative Häufigkeit $H_n(A) = \frac{h_n(A)}{n}$ eingeführt. Die Eigenschaften der relativen Häufigkeit sind:
\begin{enumerate}[label=\arabic*)]
	\item $0 \le H_n(A)$
	\item $H_n(\Omega) = 1$
	\item $A,B$ unvereinbar $\implies H_n(A \cup B) = H_n(A) + H_n(B)$ 
\end{enumerate}
\end{recall}
\begin{notation}
Die relative Häufigkeit konvergiert oft für $n \to \infty$ gegen eine feste Zahl.

In diesem Fall schreiben wir $P(A) = \lim_{n \to \infty} H_n(A)$ und sprechen von der Wahrscheinlichkeit von A. 
\end{notation} 
Außerdem motiviert Satz \ref{thm:motivation-axiome} zu einer genauen Definition:
\begin{definition}[Axiomsystem von Kolmogorov]
	Sei $(\Omega, \mathcal{A}$ ein messbarer Raum, der ein Zufallsexperiment beschreibe. Dann erfülle eine Abbildungen die Axiome:
	\begin{enumerate}[label=(\roman*)]
		\item Positivität : $0\le P(A)$ für alle $A \in \mathcal{A}$
		\item Normierung: $P(\Omega)=1$
		\item $\sigma$-Additivität: $A_i \in \mathcal{A}, i \in \N: A_i \cap A_j = \emptyset, i\neq j$ dann gilt: $P(\bigcup_{i=1}^{\infty}) = \sum_{i=1}^{\infty}P(A_i)$.  
	\end{enumerate}
	Diese Axiome implizieren die Abbildung: $P\colon \mathcal{A}\to [0,1]$. 
\end{definition}
\begin{notation}
Wir nennen:
\begin{itemize}
	\item $P$ eine Wahrscheinlichkeitsverteilung
	\item $P(A)$ als die Wahrscheinlichkeit
	\item $(\Omega, \mathcal{A}, P)$ einen Wahrscheinlichkeitsraum.
\end{itemize}
\end{notation}
\begin{theorem}Sei $(\Omega, \mathcal{A}, P)$ ein Wahrscheinlichkeitsraum dann gilt:
	\begin{enumerate}[label=\arabic*)]
		\item $P(\emptyset)=0$
		\item Für endlich viele Mengen $A_i \in \mathcal{A}, i=1,\ldots,n \in \N$ paarweise unvereinbar gilt:
				$P(\bigcup_{i=1}^{n} A_i) = \sum_{i=1}^{n}P(A_i)$
		\item $P(A) = 1- P(\overline{A})$ für alle $A \in \mathcal{A}$
		\item Für $A,B \in \mathcal{A}$ gilt: $P(A \cup B) = P(A)+P(B) -P(A \cap B)$
		\item Für $A_i \in \mathcal{A}$, paarweise unvereinbar mit $\bigcup_{i=1}^{\infty} A_i = \Omega$ gilt: $\sum_{i=1}^{\infty}P(A_i)=1$.  
	\end{enumerate}	
\end{theorem}
\begin{proof}
\begin{enumerate}[label=\arabic*)]
	\item Setze \[\begin{cases}
			A_i = \Omega &, \text{falls } i=1 \\
			A_i = \emptyset &, \text{sonst}
		\end{cases} \]
		Daraus folgt, dass $A_i, i \in \N$ paarweise unvereinbar sind.
		\begin{align*}
			&\implies 1= P(\Omega) = P(\bigcup_{i=1}^{\infty} A_i) = \sum_{i=1}^{\infty}P(A_i) = 1+ \sum_{i=2}^{\infty}P(\emptyset) \\
			&\implies P(\emptyset)=0
		\end{align*}
	\item Setze $A_i \coloneqq \emptyset$ so folgt dies direkt aus $(iii)$.
	\item $A \cup \overline{A} = \Omega$ und $A \cap \overline{A}= \emptyset$ nach $(ii), (iii)$ folgt: $1=P(\Omega)= P(A)+P(\overline{A})$ 
	\item Wir definieren: $D= A \cap B, C= B \setminus A = B \cap \overline{A}$. Dann impliziert $C$m dass $A \cup B = A \cup C$, da $A,C$ unvereinbar sind. Das heißt:\[P(A \cup B) = P(A) + P(C)\]
	Weiter gilt nun:
	\begin{align*}
	&\implies B= C \cup D \\
	&\implies P(B) =P(C \cup D) = P(C) + P(D) \\
	&\implies P(B) = P(C) +P(A \cap B) \\
	&\implies \text{Behauptung}
	\end{align*}

	\item $1 = P(\Omega) = P( \bigcup_{i=1})^{n} A_i) = \sum_{i=1}^{n} P(A_i)$ 
\end{enumerate}
\end{proof}
\begin{theorem}
	Sei $(\Omega, \mathcal{A},P) $ ein Wahrscheinlichkeitsraum.
	\begin{itemize}
		\item Für eine aufsteigende Familie $A_1 \subset A_2 \subset A_3\subset \ldots\subset A_i \in \mathcal{A}$ gilt:
			\[
				\lim_{n \to \infty} P(A_n) = P(\lim_{n \to \infty} A_n) = P\left(\bigcup_{i=1}^{\infty} A_i\right)
			\]
		\item Für eine absteigende Familie $A_1 \supset A_2 \ldots \supset A_i \in \mathcal{A} $ gilt:
			\[
				\lim_{n \to \infty} P(A_n) = P(\lim_{n \to \infty} A_n) = P\left(\bigcap_{i=1}^{\infty} A_i \right)
			\]
	\end{itemize}

\end{theorem}
\begin{proof}
Übung.
\end{proof}
Die einfachsten Modelle sind Laplace-Modelle:
\begin{theorem}[Laplace-Experiment]
	Sei $(\Omega, \mathcal{A}, P)$ ein Wahrscheinlichkeitsraum mit $|\Omega|= n \in \N$ und jedes Elementarereignis hat die gleiche Wahrscheinlichkeit: $P(\{\omega\} ) = \frac{1}{n} \forall_{\omega \in \Omega}$. Dann ist:
	\[
		P(A) = \frac{m}{n} \text{, für alle } A \in \mathcal{A} \text{ mit }|A|=m \in \N
	\]
\end{theorem}
\begin{proof}
Elementarereignisse sind unvereinbar. Es gilt:
\[
	P(A) = P(\{w \in \mathcal{A}\} ) = \sum_{i=1}^{m}\frac{1}{n}=\frac{m}{n}
\]
\end{proof}
\begin{remark}
Das Laplace-Modell sagt: $P(A)= \frac{\text{Anzahl Elementarereignisse für A}}{\text{Anzahl aller Elementarereignisse}}$ 
\end{remark}
\begin{example}
Das einmalige Werfen eines Würfels hat den Ereignisraum: $\Omega = \{1,\ldots,5\} $ und ist ein Laplace-Modell.\\ 
Für $A = \{\text{durch 3 teilbar}\} = \{3,6\} $ gilt:
\[
P(A)=\frac{|A|}{|\Omega|}=\frac{2}{6}=\frac{1}{3}
\]
\end{example}
