\section{Konvergenz von Folgen}
Eine Folge reeler Zahlen ist eine Abbildung von $\N \to \R$.
Hierbei wird eine natürliche Zahle auf eine reele Zahl geschickt.
\begin{align*}
	a_{n \in \N} &= (a_1,a_2,a_3,\ldots) \\
	a_{k \in  \Z} &= (a_k, a_{k+1},\ldots)
\end{align*}
\begin{example}
	Typische Folgen:
\begin{itemize}
	\item $\frac{1}{n}= (1,\frac{1}{2},\frac{1}{3},\ldots)$
	\item $(-1)^{n}= (1,-1,\ldots)$
	\item $2^{n}= (1,2,4,8,16,\ldots)$
	\item Fibonnacci mit $f_0 \coloneqq 0, f_1 \coloneqq 1$ und $f_n \coloneqq f_{n-1} + f_{n-2}$  
\end{itemize}
\end{example}
\begin{definition}[Konvergenz]
	Eine Folge $(a_n)_{n \in \N}$ reeler Zahlen heißt konvergent gegen $a \in \R$ falls
	\[
	\forall_{\varepsilon > 0}\exists_{N \in \N}: |a_n -a | < \varepsilon  
	\]
gilt. Äquivalent hierzu:
\begin{align*}
\forall_{\varepsilon > 0} \exists_{N} \forall_{n\ge N}: |a_n -a | < \frac{\varepsilon}{k} \\
\forall_{\varepsilon>0} \exists_{N}\forall_{n>N}: |a_n-a| \le \varepsilon   
\end{align*}
\end{definition}

