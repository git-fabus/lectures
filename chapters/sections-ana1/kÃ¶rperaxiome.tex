$\R$ ist Körper, angeordnet, archimedisch, vollständig. \\
\section{Körperaxiome}
\begin{definition}[$\R$] 
$\R$ ist eine Menge, auf der zwei Verknüpfungen definiert sind:
\begin{align*}
+ &\colon \R \times \R \to \R \\
\cdot  &\colon \R \times \R \to \R 
\end{align*}
Dabei wird jedem Paar $(x,y)$ mit $x,y \in \R$ ein Element $x+y \in \R$ (bzw $x\cdot y \in \R$ zu geordnet.  
\end{definition}
\begin{remark}
Die zwei Verknüpfungen sind die zwei Grundrechenarten "plus" und "mal". 
Jedoch müssen wir die typischen Rechenregeln, die wir bereits kennen erst anhand der Körperaxiome zeigen.
\end{remark}
\begin{definition}[Körper]
	$\K$ ist ein Körper falls folgende Axiome gelten:
	Es sei $a,b,c \in \K$. \\
	Für die Addition:
	\begin{description}
		\item [A1] $\forall_{a,b,c}: a+(b+c) = (a+b)+c$
		\item [A2] $\forall_{a,b}: a+b = b+a $
		\item [A3] $\exists_{0 \in \K}: a+0=a $
		\item [A4] $\forall_{a}\exists_{-a \in \K}: a+(-a)=0$
	\end{description}
	Analog weiter für die Multiplikation:
	\begin{description}
		\item [M1] $\forall_{a,b,c}: (a\cdot b) \cdot c = a \cdot (c \cdot b) $
		\item [M2] $\forall_{a,b}: a \cdot b = b \cdot a $
		\item [M3] $\exists_{1 \in \K}\forall_{a}: 1\neq 0 \wedge 1\cdot a=a$
		\item [M4] $\forall_{a\neq 0}\exists_{a^{-1} \in \K}: a\cdot a^{-1}=1 $  
	\end{description}
	Außerdem gilt das Distributivgesetz:
	\begin{description}
		\item [D] $\forall_{a,b,c}: (a+b)\cdot c = a\cdot c + b\cdot c $
	\end{description}
\end{definition}
\begin{notation}
Vereinfachend schreiben wir auch:
\begin{itemize}
	\item $a\cdot b = ab$
	\item $a^{-1}=\frac{1}{a}$
	\item $a\cdot b^{-1}= \frac{a}{b}$ 
\end{itemize}
\end{notation}
\subsubsection*{Rechenregeln für Körper}
\begin{remark}
Die Eigenschaften A1-A4 besagen außerdem, dass $(\R,+)$ eine Gruppe ist.
\end{remark}
\begin{enumerate}
	\item Eindeutigkeit der $0$
		\begin{proof}
		$0^{*}= 0^{*}+0 = 0+0^{*}=0$ 
		\end{proof}
	\item Das Negativ $-x$ ist eindeutig.
		\begin{proof}
		Seien $x'', x'$ für $(x'+x=0) \wedge (x+x'' =0) $ gilt. \\
		$\implies x'=x'+0 = x'+x(-x) = x'' +x+(-x) = x''$ 
		\end{proof}
	\item $-0 = 0$
		\begin{proof}
		$0= 0+(-0) = (-0)+0= -0$
		\end{proof}
	\item $\forall_{x}: -(-x) = x $
		\begin{proof}
		$0=x+(-x)= (-x)+x$ und Eindeutigkeit des Negativen folgt: $x=-(-x)$   
		\end{proof}
	\item $\forall_{a,b,c \in \R}: a+b= c \iff a = c-b$
	\item $\forall_{a,b \in \R}: -(a+b) = (-a) + (-b) $ 
\end{enumerate}
\begin{remark}
Die Eigenschaften M1-M4 besagen, dass $\R \setminus \{0\}, \cdot )$ auch eine Gruppe ist. Demnach gelten die Eigenschaften a-f eigentlich fast Analog für die Multiplikation.
\end{remark}
Wegen der Distributivität folgt außerdem:
\begin{enumerate}[resume]
	\item $\forall_{a,b,c \in \R}: (a+b) \cdot c = c\cdot a + c\cdot b $ 
	\item $\forall_{a \in \R}: x\cdot 0=0 $
		\begin{proof}
		$x\cdot 0=x\cdot 0+x\cdot 0 = x \cdot (0+0) = x\cdot 0$ 
		\end{proof}
	\item $\forall_{a,b \in \R}: a \cdot b = 0 \iff (x=0) \vee (y=0)$
		\begin{proof}
		Hinrichtung folgt aus h. \\
		\emph{Annahme:} $xy=0$ und $x\neq 0$. Zu zeigen: $y=0$\\
		Mit $y=0\cdot x^{-1}$ und h folgt: $y=0$ 
		\end{proof}
\end{enumerate}
\begin{remark}
Jede Menge $K$ mit Verknüpfungen $+$ und $\cdot$ für die Axiome gelten heißt Körper.  
\end{remark}

\begin{example}
	Dies sind Körper:
\begin{itemize} 
	\item $\R, \Q, \C$
	\item $\mathbb{F}_p = \{0,1,\ldots,p\} $ mit Addition / Multiplikation $\mod p$ 
	\item $\mathbb{F}_2 = \{0,1\}$
\end{itemize}
\end{example}
\begin{enumerate}[resume]
	\item $\forall_{x}: -x = (-1)x $
		\begin{proof}
		$x+(-1)x=1x+(-1)x=(1+(-1))x=0x=0$ 
		\end{proof}
	\item $(-x)\cdot (-y) = xy$
		\begin{proof}
			$(-x)\cdot (-y) = (-x)(-1)y= ((-1)\cdot (-x))y=(-(-x)\cdot y)=xy$ 
		\end{proof}
\end{enumerate}
Im Allgemein gilt:
\begin{itemize}
	\item Assoziativ: \\
		$\forall_{n \in \N}, \forall_{x_1,\ldots,x_{n} \in \R} : x_1+\ldots+x_{n} = x_1+ (x_2 +(x_3\ldots)x_{n-1} + x_{n}$
	\item Kommutativ: \\
		$\forall_{n \in  \N}, \forall_{x_1,\ldots,x_{n} \in \R} : x_1+\ldots+x_{n}= x_{n}+x_{n-1}+ \ldots + x_1$
\end{itemize}
\begin{theorem}[Doppelsummen]
\[
	\forall_{m,n \in  \N}, \forall_{a_{ij} \in  \R}: \sum_{i=1}^{m} \left( \sum_{j=1}^{n}a_{i,j} \right) = \sum_{j=1}^{n} \left( \sum_{i=1}^{m}a_{i,j} \right)
\]
\end{theorem}
\begin{proof}
	\begin{align*}
	\sum_{i=1}^{m}\sum_{j=1}^{n}a_{i,j} &= \left( \sum_{j=1}^{n}a_{1,j} + \ldots + \sum_{j=1}^{n}a_{n,j} \right) \\
					    &= \left( \sum_{i=1}^{m}a_{i,1} + \ldots + \sum_{i=1}^{m}a_{i,m} \right) \\
					    &=\sum_{j=1}^{n}\sum_{i=1}^{m}a_{i,j}
\end{align*}
\end{proof}
Es gilt ebenfalls das Distributivgesetz für Summen.
\begin{theorem}[Distributiv]
	\[
		\left( \sum_{i=1}^{m}x_i \right) \cdot \left(\sum_{j=1}^{n}y_{i} \right) = \sum_{i=1}^{m}\sum_{j=1}^{n}x_{i}y_{j}
\]
\end{theorem}
\begin{definition}[Potenzen]
$\forall_{n \in \N_0}, \forall_{x \in \R} : x^{n} \in \R $ \\
Weiter definieren wir: \\
$x^{0} \coloneqq 1$ und $x^{n+1} \coloneqq x \cdot x^{n}, \forall_{n \in \N}$. \\
Außerdem gilt für $x \neq 0: x^{-n}\coloneqq (x^{-1})^{n}$
\end{definition}
Daraus folgt:
\begin{enumerate}[resume]
	\item $\forall_{m,n \in  \Z}, \forall_{x,y \in \R}: x^{m}x^{n}= x^{m+n}$, $\left(x^{m}\right)^{n}= x^{mn}$ und $x^{n}y^{n}= (xy)^{n}$   
\end{enumerate}
\subsection{Anordnungsaxiome}
\begin{description}
	\item [Trichotomie] $\forall_{x \in \R}$ gilt genau eine der folgende Aussagen: $x>0, x=0 , x<0$
	\item [Abgeschlossenheit] $\forall_{x,y \in \R}: x>0$ und $y>0 \implies x+y >0$ und $xy>0$   
\end{description}
\begin{notation}
\begin{itemize}
	\item $x>y \iff x-y >0$
	\item $x<y \iff x-y < 0$
	\item $x \ge y \iff x>y$ oder $y=x$
	\item $x\le y \iff x<y$ oder $x=y$  
\end{itemize}
\end{notation}
Daraus ergeben sich weitere Konsequenzen:
\begin{enumerate}[resume]
	\item $\forall_{x,y}: x<y $ oder $x=y$ oder $x>y$
	\item Transitivität: $x<y$ und $y<z \implies x<z$
		\begin{proof}
		$x<y$ und $y<z \implies y-x>0$ und $z-y > 0 \implies z-x > (z-y) + (y-x) >0$   
		\end{proof}
	\item Translation-Invarianz: $x<y \implies \forall_{z}: x+z < y+z$
		\begin{proof}
			$(y+z)-(x-z) = y-x >0$ 
		\end{proof}
	\item Spiegelung: $x<y \iff (-x) > (-y)$
		\begin{proof}
			$(-x)-(-y) = y-x > 0$ 
		\end{proof}
	\item $x<y$ und $u < v \implies x+u < y+v$
	
	\item $x<y$ und $t>0 \implies tx <ty$
	\item $x<y$ und $t<0 \implies tx<ty$
	\item $0 \le x < y$ und $0  \le x < y$ und $0 \le  u < v \implies xu < yv$
	\item $\forall_{x\neq 0}: x>0 \iff x^{-1} >0 $ 
	\item $0<x<y \implies x^{-1}> y^{-1}$
	\item $\forall_{x \neq 0}: x^{2}>0$ insbesondere gilt: $1>0$
		\begin{proof}
		Falls $x>0$ gilt: Abgeschlossenheit der Multiplikation\\
		Sonst: $x^{2}= (-x) \cdot (-x) > 0 $ ebenfalls gilt die Abgeschlossenheit der Multiplikation
		\end{proof}
\end{enumerate}

\subsection{Das Archimedische Axiom}
\[
\forall_{x \in \R}\exists_{n \in \N}: n>x  
\]
Daraus folgt:

\begin{enumerate}[resume]
	\item $\forall_{\varepsilon >0}\exists_{n \in  \N}: \frac{1}{n} < \varepsilon$
	\item $\forall_{x \in  \R} \exists!_{n \in  \N}: n \le x < n+1 $ 
\end{enumerate}

\begin{theorem}[Bernulische Ungleichung]
\[
\forall_{x \in  \R, x \ge -1} \forall_{n \in  \N}: (1+x)^{n} \ge 1 + nx   
\]
\end{theorem}
\begin{proof}
durch Induktion.
\begin{itemize}[label=$\lozenge$, itemsep=2ex]
	\item \underline{IA $n=1$}
		\[
		1+x \ge 1+x
		\]
	\item \underline{ $n \to n+1$}
		da $A(n)$ gilt: \\
		Es bleibt zu Zeigen: $(1+x)^{n+1} \ge 1+ (n+1)x$.\\
		Nach der Induktionsannahme gilt:
		\begin{align*}
			(1+x)(1+nx) &\ge 1 + (n+1)x \\
			1+nx+x+nx^{2} &\ge  1+nx+x
\end{align*}	
\end{itemize}
\end{proof}
\begin{corollary}
	\begin{itemize}
		\item $\forall_{a \in \R, a>1} \forall_{K \in \R } \exists_{n \in \N}: a^{n} > K $
	\item $\forall_{a \in \R, 0<a<1} \forall_{\varepsilon >0} \exists_{n \in  \ N} : a^{n} < \varepsilon  $  
	\end{itemize}
\end{corollary}
\begin{proof}
\begin{itemize}
	\item Bernulische Ungleichung mit $x=a\cdot 1 > 0$ 
	\item Analog zu (i) nur mit $\frac{1}{a}$ statt $a$ und $K=\frac{1}{\varepsilon}$   
\end{itemize}
\end{proof}

\begin{definition}[Absolut-Betrag]
\[
x \in \R : |x| \coloneqq \begin{cases}
	x &, \text{falls } x\ge 0 \\
	-x &, \text{falls } x<0
\end{cases} \]
\end{definition}
\begin{definition}[Maximum]
\[
\max(x,y) \coloneqq \begin{cases}
	x &, x\ge y \\
	y &, \text{sonst}
\end{cases}
\]
\end{definition}
\begin{theorem}
	Für den Betrag $|•|$ gilt:
\begin{itemize}
	\item $\forall_{x}: |x| \ge 0 $ 
	\item $\forall_{x,y}: |xy| = |x| \cdot |y| $
	\item $\forall_{x,y}: |x+y| \le |x| + |y| $ 
\end{itemize}
\end{theorem}
\begin{proof}
Wir betrachten nur Aussage 3, da die anderen Aussagen trivial sind: \\
Weil $x\le |x|$ gilt:
\begin{align*}
	&x+y \le |x| + |y| \\
	&\implies -x\le |x| \\
	&-(x+y) = (-x) + (-y) \le (x)+(y) \\
	&\implies |x+y| \le |x| + |y|
\end{align*}
\end{proof}
