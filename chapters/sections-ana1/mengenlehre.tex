\section{Mengenlehre}
\begin{definition}[Menge nach Cantor] 
	Eine Menge ist die Zusammenfassung bestimmter wohlunterschiedener Objekte, welche die Elemente der Menge genannt werden, unserer Anschauung oder unseres Denkens zu einem Ganzen.
\end{definition}
\begin{example}
Die typischen Zahlenmengen kennt man bereits:
\begin{itemize}
	\item Menge der natürlichen Zahlen $\N$
	\item Menge der rationalen Zahlen $\Q$ 
	\item Menge der reelen Zahlen $\R$
\end{itemize}
\end{example}
\begin{notation}Wir bezeichnen:
\begin{table}[htpb]
	\centering
	\begin{tabular}{c|c}
		Symbole & Gesprochen \\
		\hline
		$x \in M$ & x Element der Menge M \\
		$M \subset N$ & M Teilmenge von N \\
		$M \subseteq N$ & M strikte Teilmenge von N \\
		$\O = \{ \} $ & leere Menge \\
		$M \cap N$ & Schnittmenge von M und N \\
		$M \cup N$ & Vereinigung von M und N \\
		$M \setminus N$ & M ohne N \\
		$N^{c}$ & Komplement der Menge \\
		$M \cap N = \{ \} $ & $\implies$ M und N sind disjunkt
	\end{tabular}
\end{table}
\end{notation}
\begin{remark}[Grenzen der naiven Mengenlehre]
	Nicht erlaubt ist die Konstruktion der Menge aller Mengen, die sich selbst als Element enthalten.
\end{remark}
